
\usepackage{graphicx}


\usepackage[utf8]{inputenc} %Together with 'spanish' package, allows you to write accents 

%This package allows the use of accents, the parameter 'es-tabla' writes Tabla instead of 'Cuadro', the parameter es-noindentfirst makes that the first line after each section and subsection is not indented
\usepackage[spanish,activeacute,es-tabla,es-noindentfirst]{babel}

%This package allows you among other things, use as option [H] in tables and figures, which set the object just where you put in the source code.
\usepackage{float}

%This package allows you to write pseudocode, you should read the docummentation to use it properly 
\usepackage{algorithm2e}

%Packages to write math
\usepackage{amssymb}
\usepackage{amsmath}

%permite el formato de múltiple citación agrupada dentro de los corchetes
\usepackage{cite} 

\usepackage{times}
\usepackage{color}

%Package that help to write text that will appear like you write in the final documment
\usepackage{verbatim}

%This package allows you to configure tables and figures
\usepackage{caption}

%Package to customize formats to the tables
\usepackage{booktabs}

%Package to control hiperlinks
\usepackage[breaklinks=true]{hyperref}
\hypersetup{
	colorlinks=true,
	linkcolor=blue,
	filecolor=magenta,      
	urlcolor=blue,
	citecolor=cyan,
}

%Package to stablish the margins of the documment
\usepackage{vmargin}

%A0, A1, ..., A9, B0, B1, ..., B9, C0, ..., C9, USletter, USlegal, and USexecutive
\setpapersize{A4}
%\setmarginsrb{hleftmargini}{htopmargini}{hrightmargini}{hbottommargini}%
%{hheadheighti}{hheadsepi}{hfootheighti}{hfootskipi}

\setmarginsrb{30mm}{25mm}{30mm}{25mm}{6mm}{7mm}{5mm}{15mm}

%También puede utilizar esta sintaxis para establcer los márgenes con el paquete {vmargin}
%\setmargins{3.0cm}       % margen izquierdo
%{1.5cm}                        % margen superior
%{14.5cm}                      % anchura del texto
%{23.42cm}                    % altura del texto
%{10pt}                           % altura de los encabezados
%{1cm}                           % espacio entre el texto y los encabezados
%{0pt}                             % altura del pie de página
%{2cm}         

%Estos paquetes se utilian para escribir psudocódigo, sin embargo en estos momentos se está utilizando el paquete {algorithm2e}
%\usepackage{algpseudocode}
%\usepackage{algorithmicx}
%\usepackage{algorithm}

\definecolor{gray97}{gray}{.97}
\definecolor{gray75}{gray}{.75}
\definecolor{gray45}{gray}{.45}

%Packages to write code in various programming languajes, please, see the docummentation 
\usepackage{listings}
\usepackage{listingsutf8}
\lstset{frame=Ltb,
	framerule=0pt,
	aboveskip=0.5cm,
	framextopmargin=3pt,
	framexbottommargin=3pt,
	framexleftmargin=0.4cm,
	framesep=0pt,
	rulesep=.4pt,
	backgroundcolor=\color{gray97},
	rulesepcolor=\color{black},
	%
	stringstyle=\ttfamily,
	showstringspaces = false,
	basicstyle=\small\ttfamily,
	commentstyle=\color{blue},
	keywordstyle=\bfseries,
	%
	numbers=left,
	numbersep=15pt,
	numberstyle=\tiny,
	numberfirstline = false,
	breaklines=true,
}
\lstnewenvironment{listing}[1][]
{\lstset{#1}\pagebreak[0]}{\pagebreak[0]}

\lstdefinestyle{consola}
{basicstyle=\scriptsize\bf\ttfamily,
	backgroundcolor=\color{gray75},
}
\lstdefinestyle{C}
{language=C,
}
% Used for displaying a sample figure. If possible, figure files should
% be included in EPS format.
%
% If you use the hyperref package, please uncomment the following line
% to display URLs in blue roman font according to Springer's eBook style:
\renewcommand\UrlFont{\color{blue}\rmfamily}

%\renewcommand\Algorithmname{Algoritmo}
\renewcommand\examplename{Ejemplo}
\renewcommand\exercisename{Ejercicio}
\renewcommand\figurename{Fig.}
\renewcommand\keywordname{{\bf T\'erminos Clave:}}
\renewcommand\indexname{Index}
\renewcommand\lemmaname{Lema}
\renewcommand\contriblistname{Lista de colaboradores}
\renewcommand\listfigurename{Lista de Figuras}
\renewcommand\listtablename{Lista of Tablas}
\renewcommand\mailname{{\it Correspondencia para\/}:}
\renewcommand\noteaddname{Note added in proof}
\renewcommand\notename{Nota}
\renewcommand\partname{Parte}
\renewcommand\problemname{Problema}
\renewcommand\proofname{Demostración}
\renewcommand\propertyname{Propiedad}
\renewcommand\propositionname{Proposici\'on}
\renewcommand\questionname{Pregunta}
\renewcommand\remarkname{Remark}
\renewcommand\seename{Ver}
\renewcommand\solutionname{Soluci\'on}
\renewcommand\theoremname{Teorema}